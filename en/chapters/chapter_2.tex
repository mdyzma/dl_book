% -*- root: ../main.tex -*-
\chapter{Mathematical optimization}
\label{chap:optimization}


\section{Guess \& check}


\section{Hill climbing}

Let’s think how we could guide our search so that it does not check as
much of bad solutions. What kind of information we could use? Again we
ask a reader to think on his own about all of the questions that will be
asked going forward.

One method which could help us is better choice of a points to be checked.
If the space is continuous we can use our knowledge of points lying nearby
to approximate the value of point in question. We could think of this
problem as climbing a hill in the fog. We look for the steepest direction
which we can see and move in this direction.The Hill climbing algorithm keeps track of one currently best solution,
checks the points nearby and moves in the direction of the biggest
improvement, as long as any improvement can be made.

In n dimensions one should check 2*n directions in orthogonal way
(which means at right angle but for every number of dimensions)
backward and forward the number line by some vector (some displacement
in space), before changing current best point. This problem seems very
obvious, as number of dimension grows also the number of possibilities
we must check grows enormously. Also we don’t necessarily move in a
best directions since it can be that the best directions lies in between the
directions we checked. Finally we do not use any of so called gradient
information.


\section{Gradient descent}


\section{ Gradient descent + step size}



\section{ Gradient descent + momentum}


\section{Stochastic gradient descent}